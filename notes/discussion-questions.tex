\documentclass[10pt]{article}
\usepackage[a4paper, total={6.5in, 11in}, includefoot, heightrounded]{geometry}
\usepackage{adjustbox}
\usepackage{array}
\usepackage{color}

\title{Structured Representations in Standards Documents: Discussion Questions}
\date{\today}

\newcommand{\todo}[1]{\textbf{\textcolor{red}{To do -- #1}}}
\newcommand{\question}[1]{\vspace{4mm}\noindent\textbf{Q:} \textit{#1}}

\begin{document}
\maketitle

\section{Existing Structured Representations in IETF Documents}

\subsection{Examples}

\subsubsection*{ABNF}
\subsubsection*{XML}
\subsubsection*{ASN.1}
\subsubsection*{C code}
\subsubsection*{JSON}
\subsubsection*{CBOR}
\subsubsection*{TLS 1.3 Presentation Language}

\todo{Add example snippets}

\subsection{Questions}

\question{Have you encountered any of the above structured description formats in IETF documents?}

\question{If so, did they improve the utility of the document? In what way?}

\question{Have you authored documents that include any of the above?}

\question{If not, why not?}

\question{Do you think that the current level of adoption of these formats in standards documents is sufficient?}

\question{If not, what do you see as the barriers to adoption?}

\section{Parsable Protocol Standards}

In most standards documents, the syntax of the protocol is specified using an ASCII art
packet header diagram, showing the alignment and order of the fields, followed by a prose
description of each field.

One of the key limitations of the existing approach is that it limits the use of tooling,
where the parsing of English prose is non-trivial. Tooling would be viable with the use of
a more formal, structured approach to syntax specification. The benefits of tooling vary
with the amount of information that can be captured, but at its simplest, would allow for
a parser to be generated for the specified protocol. 

\question{What are the benefits of the existing approach?}

\question{What are the limitations (beyond parsability)?}

\todo{Get at the balance between formal specification and usability/adoption: need something
people can/will use, but that allows for at least some tooling to be developed}

\bibliographystyle{abbrv}
\bibliography{discussion-questions}

\end{document}