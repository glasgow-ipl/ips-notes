\documentclass[10pt,twocolumn,a4paper]{article}
\usepackage[l2tabu,orthodox]{nag}
\usepackage[utf8x]{inputenc}
\usepackage[british]{babel}
\usepackage[babel=true]{microtype}
\usepackage{amsmath}
\usepackage[all]{onlyamsmath}
\usepackage{newtxtext}
\usepackage{newtxmath}
\usepackage{upquote}
\usepackage{graphicx}
\usepackage{url}
\usepackage[caption=false]{subfig}
\usepackage{booktabs}
\usepackage{bytefield}
\usepackage{listings}
\usepackage{algorithm}
\usepackage{algpseudocode}
\usepackage{color}
\usepackage[cm]{fullpage}
\usepackage{no-par-indent}
\usepackage{alltt}
\lstset{basicstyle=\footnotesize\ttfamily,breaklines=true}
\frenchspacing
\newcommand{\todo}[1]{\textit{\textcolor{red}{[To do: #1]}}}
\newcommand{\code}[1]{\texttt{#1}}

%==================================================================================================
\begin{document}
\title{The Glasgow Packet Language: Type System and Execution Model}
\author{
  Stephen McQuistin\\University of Glasgow
\and
  Vivian Band\\University of Glasgow
\and
  Colin Perkins\\University of Glasgow
}
\date{\today}
\maketitle
%==================================================================================================
\begin{abstract}

  This memo defines the intermediate representation used in the
  implementation of the Glasgow Packet Language. It also describes
  the execution model used to parse protocols described in that
  language.

\end{abstract}
%==================================================================================================
\section{Introduction}

% Paragraph 1: Motivation. At a high level, what is the problem area you
% are working in and why is it important? It is important to set the larger
% context here. Why is the problem of interest and importance to the larger
% community?



% Paragraph 2: What is the specific problem considered in this paper? This
% paragraph narrows down the topic area of the paper. In the first
% paragraph you have established general context and importance. Here you
% establish specific context and background.

In order to generate a parser for a protocol, it's necessary to describe
the format of protocol data units (PDUs).
These can be described by a set of types that represent the objects to be
parsed, along with constraints on the parsed values.
If there are multiple ways of describing these types, it's necessary to
have a common \emph{intermediate representation} into which all those input
formats can be converted, and from which parsers can be generated.

% Paragraph 3: "In this paper, we show that...". This is the key paragraph
% in the introduction - you summarize, in one paragraph, what are the main
% contributions of your paper, given the context established in paragraphs
% 1 and 2. What's the general approach taken? Why are the specific results
% significant? The story is not what you did, but rather:
%  - what you show, new ideas, new insights
%  - why interesting, important?
% State your contributions: these drive the entire paper.  Contributions
% should be refutable claims, not vague generic statements.

We describe such an intermediate representation in this memo.
It specifies a set of internal types used by the parsers, and constructors
for representable types that describe the PDUs to be parsed.
The execution model of the parsers is also described.

% Paragraph 4: What are the differences between your work, and what others
% have done? Keep this at a high level, as you can refer to future sections
% where specific details and differences will be given, but it is important
% for the reader to know what is new about this work compared to other work
% in the area.



% Paragraph 5: "We structure the remainder of this paper as follows." Give
% the reader a road-map for the rest of the paper. Try to avoid redundant
% phrasing, "In Section 2, In section 3, ..., In Section 4, ... ", etc.


%==================================================================================================
\section{Terminology}

The key words ``MUST'', ``MUST NOT'', ``REQUIRED'', ``SHALL'', ``SHALL
NOT'', ``SHOULD'', ``SHOULD NOT'', ``RECOMMENDED'', ``NOT RECOMMENDED'',
``MAY'', and ``OPTIONAL'' in this document are to be interpreted as
described in \cite{RFC2119,RFC8174} when, and only when, they appear
in all capitals, as shown here.

\emph{Runtime} refers to software that supports the type system that this
document describes. The runtime instantiates a set of primitive types, provides
functions for the construction of other types, and synthesises the primitive and
constructed types into a state that is suitable for the generation of parser or
serialisation code.

\emph{Implementation} refers to code that is generated, with support from the
runtime, to parse or serialise the objects that the type system describes.

%==================================================================================================
\section{Type System}

Types can be \emph{internal} or \emph{representable}. An internal type is an
artefact of the runtime, and cannot be parsed or serialised. A representable
type describes something that can be parsed or serialised.

The runtime provides a set of \emph{type constructors} that can be used to
create different kinds of type, and instantiates a set of primitive types.

Each type has a \emph{kind}, a \emph{name}, and a \emph{value}.
A type implements one or more \emph{traits}.
Traits define \emph{methods} that can operate on instances of that type.

A type name is formed of upper- and lower-case ASCII letters, digits, and
dollar signs (\code{A-Za-z0-9\$}). It must begin with an upper case letter
(\code{A-Z}). It is an error to define two types with the same name.

%--------------------------------------------------------------------------------------------------
\subsection{Traits}
\label{sec:traits}

A \emph{trait} defines a named collection of methods. A type may implement
one or more traits. A type that implements a trait gains an implementation
of each of the trait's methods.

A trait definition specifies the name of the trait and defines the methods
it provides.

A trait name is comprised of upper and lower case ASCII letters, digits, or
dollar signs (\code{A-Za-z0-9\$}).  It MUST begin with an upper case letter
(\code{A-Z}). Trait names occupy a different namespace to type names.

A method definition specifies its name, parameters, and return type. Methods
take one or more parameters. Each parameter has a name and a type. The first
parameter of a method MUST have the name \code{self} and MUST have the type
parameterised as \code{Self}. Type parameter \code{Self} refers to the type
that implements the trait. Other parameters, if any, MUST have names that are
unique within the method definition.

Types used in method definitions MAY be concrete types (i.e., types
specified when the trait is defined) or type parameters. Concrete types for
any type parameters MUST be provided when a type that implements the trait
is constructed.

Method names and parameter names are formed of upper and lower case ASCII
letters, digits, dollar signs, or underscores (\code{A-Za-z0-9\$\_}). They
MUST begin with a lower case letter (\code{a-z}).
It is an error to define two methods that have the same name within a given
trait.

When a method is called on an instance of a type that implements a trait,
that instance will be passed as the first parameter of the method.

It is not possible for further traits to be defined beyond those defined by the
runtime.

The runtime defines the following traits:
\begin{itemize}
  \item The \code{Value} trait defines the following two methods that can
    be used to get and set the value of an instance of a type:
    \begin{itemize}
      \item \texttt{get(self: Self) $\rightarrow$ Self}
      \item \texttt{set(self: Self, value: Self) $\rightarrow$ Nothing}
    \end{itemize}
    The syntax \emph{name(parameters) $\rightarrow$ return\_type} is used to
    describe a method. The parameters are a list of \emph{name: type}
    pairs. \code{Nothing} is a primitive representable type defined in
    Section~\ref{sec:representable-primitive}.

  \item The \code{Sized} trait defines a method that can be used to get the
    size, in bits, of an instance of a type:
    \begin{itemize}
      \item \texttt{size(self: Self) $\rightarrow$ Number}
    \end{itemize}
    \code{Number} is a primitive internal type defined in
    Section~\ref{sec:internal-primitive}.
    All representable types MUST implement the \code{Sized} trait.
    \code{size} MAY return an expression (Section~\ref{sec:expressions}) that
    evaluates to a \code{Number}.

  \item The \code{IndexCollection} trait defines methods that can be used
    to get and set the values of elements of an array-like data structure,
    which has multiple, numerically indexed elements:
    \begin{itemize}
      \item \texttt{get(self: Self, index: Number) $\rightarrow$ \emph{T}}
      \item \texttt{set(self: Self, index: Number,\\\hspace*{0.75cm}value: \emph{T}) $\rightarrow$ Nothing}
    \end{itemize}
    Type parameter \texttt{\emph{T}} MUST NOT be \code{Self} and MUST NOT be
    \code{Nothing}. The type parameterised by \texttt{\emph{T}} MUST be
    specified when the trait is implemented.

  \item The \code{Equality} trait defines methods that can be used to
    compare to instances of a type for equality:
    \begin{itemize}
      \item \texttt{eq(self: Self, other: Self) $\rightarrow$ Boolean}
      \item \texttt{ne(self: Self, other: Self) $\rightarrow$ Boolean}
    \end{itemize}
    \code{Boolean} is a primitive internal type defined in
    Section~\ref{sec:internal-primitive}.

  \item The \code{Ordinal} trait defines methods that can be used to
    compare the values of instances of a type:
    \begin{itemize}
      \item \texttt{lt(self: Self, other: Self) $\rightarrow$ Boolean}
      \item \texttt{le(self: Self, other: Self) $\rightarrow$ Boolean}
      \item \texttt{gt(self: Self, other: Self) $\rightarrow$ Boolean}
      \item \texttt{ge(self: Self, other: Self) $\rightarrow$ Boolean}
    \end{itemize}

  \item The \code{BooleanOps} trait defines methods that can be used to
    perform boolean operations on, and between, instances of a type:
    \begin{itemize}
      \item \texttt{and(self: Self, other: Self) $\rightarrow$ Boolean}
      \item \texttt{ or(self: Self, other: Self) $\rightarrow$ Boolean}
      \item \texttt{not(self: Self) $\rightarrow$ Boolean}
    \end{itemize}

  \item The \code{ArithmeticOps} trait defines methods that can be used to
    perform arithmetic operations on instances of a type:
    \begin{itemize}
      \item \texttt{plus(self: Self, other: Self) $\rightarrow$ Self}
      \item \texttt{minus(self: Self, other: Self) $\rightarrow$ Self}
      \item \texttt{multiply(self: Self, other: Self)\\\hspace*{0.5cm}$\rightarrow$ Self}
      \item \texttt{divide(self: Self, other: Self) $\rightarrow$ Self}
      \item \texttt{modulo(self: Self, other: Self) $\rightarrow$ Self}
    \end{itemize}

  \item The \code{NumberRepresentable} trait defines a method that can be
    used to represent a type as an \code{Number} type:
    \begin{itemize}
      \item \texttt{to\_number(self: Self) $\rightarrow$ Number}
    \end{itemize}

  \item The \code{TypeEquality} trait defines a method that can be used to
    determine if two type instances are of the same type:
    \begin{itemize}
      \item \texttt{is\_type(self: Self, other: \emph{T})\\\hspace*{0.5cm}$\rightarrow$ Boolean}
    \end{itemize}
    where \code{is\_type} returns \code{true} if \code{Self} and \code{\emph{T}}
    are the same type, and \code{false} otherwise.
\end{itemize}

%--------------------------------------------------------------------------------------------------
\subsection{Internal types}

The runtime defines two primitive internal types (\code{Boolean} and
\code{Number}) and type constructors for three kinds of internal type
(\code{Function}, \code{Protocol}, and \code{Context}).

% - - - - - - - - - - - - - - - - - - - - - - - - - - - - - - - - - - - - - - - - - - - - - - - - -
\subsubsection{Primitive type}
\label{sec:internal-primitive}

The runtime defines two primitive internal types:

\begin{itemize}
  \item The \code{Boolean} type is a boolean value, either \code{true} or
    \code{false}.  It implements the traits \code{Value}, \code{Equality}
    (two \code{Boolean}s are equal if they have the same value), and
    \code{BooleanOps}.

  \item The \code{Number} type is a positive integral value. It implements the
    traits \code{Value}, \code{Equality}, \code{Ordinal}, and
    \code{ArithmeticOps}. The size of the \code{Number} type is implementation defined.
\end{itemize}

% - - - - - - - - - - - - - - - - - - - - - - - - - - - - - - - - - - - - - - - - - - - - - - - - -
\subsubsection{Functions}
\label{sec:functions}

A \code{Function} type can be constructed to represent
the signature of a function to be provided by the implementation of
the protocol. The code comprising the body of the function is not captured by
the intermediate representation.

The \code{Function} type constructor is parameterised by the \emph{name} of
the function, the list of \emph{parameters} that the function takes (each
with its own name and type), and the function's \emph{return type}.
The return type \code{Nothing} is used for functions that return no value.

Function names and parameter names are formed of upper and lower case ASCII
letters, digits, dollar signs, or underscores (\code{A-Za-z0-9\$\_}), and
MUST begin with a lower case letter (\code{a-z}).
Function names are defined in the global type namespace.
Each parameter within a function definition MUST have a unique name.

% - - - - - - - - - - - - - - - - - - - - - - - - - - - - - - - - - - - - - - - - - - - - - - - - -
\subsubsection{The Parsing Context}
\label{sec:context}

A \code{Context} type
represents a sequence of fields, possibly of different types, that can be
accessed in the parsing of other types, where needed.

The \code{Context}
constructor is parameterised by the details of the \emph{fields} it includes.

The \code{Context} holds internal state for the parser, and does not
represent a PDU, or any part of a PDU, of the protocol being parsed.
The \code{Context} is defined along with the other \emph{type definitions}
in a \code{Protocol} (defined in Section~\ref{sec:protocol}), however, since it
is specific to a protocol.

Each field has a name that is formed of upper and lower case ASCII letters,
digits, dollar signs, or underscores (\code{A-Za-z0-9\$\_}). The field name
MUST begin with a lower case letter (\code{a-z}). Field names must be unique
within the \code{Context}.

Each field has a type. That type MUST be a previously defined representable
type (primitive representable types and type constructors for representable
types are defined in Section~\ref{sec:representable}).

% - - - - - - - - - - - - - - - - - - - - - - - - - - - - - - - - - - - - - - - - - - - - - - - - -
\subsubsection{Protocols}
\label{sec:protocol}

A \code{Protocol} type can be constructed to represent
the types and protocol data units (PDUs) that form a protocol.

The \code{Protocol} type constructor is parameterised by the \emph{name} of the
protocol, a set of \emph{type definitions} for types used by the protocol parser,
and the list of type names of \emph{PDUs} used in the protocol.

Any type constructor for a representable type (defined in
Section~\ref{sec:representable}) or a function type MAY be included in the set
of type definitions. In addition, the set MUST include one context type
constructor. This context MUST include a field named ``\code{data\_size}'' with
the type \code{Number}. As described in Section~\ref{sec:parser-model}, the
implementation will set this field to the size of the current data unit being
parsed or serialised.

The list of PDU type names MAY be empty, although this represents a
protocol that is useless. PDU type names can be type names of any representable
type whose constructor is part of the set of type definitions of the protocol.

%--------------------------------------------------------------------------------------------------
\subsection{Representable types}
\label{sec:representable}

The runtime specifies two primitive representable types (\code{Nothing} and
\code{DataUnit}). In addition, the runtime provides type
constructors for \code{BitString}, \code{Option}, \code{Array}, \code{Struct},
and \code{Enum} types. Finally, a mechanism by which new types can be derived
from existing representable types is also defined.

The base representable type is a \code{DataUnit} (defined in
Section~\ref{sec:representable-primitive}). All data for parsing or
serialisation is initially represented within the language as a \code{DataUnit}.

\code{BitString} and \code{Option} types cannot be initialised from another
type. All other representable types MUST be associated with a single initialisation
function, with the following signature:

\begin{center}
  \texttt{init(from: \emph{T}) $\rightarrow$ Option<Self>}
\end{center}

where \code{\emph{T}} is any previously defined representable type and MUST NOT
be \code{Nothing}, and \code{Self} refers to the type that is being instantiated.
A type's \code{init} function constructs and initialises an instance of that
type, from an instance of the type \code{\emph{T}}. These functions exist within
the namespace for all functions, as described in Section~\ref{sec:functions}.

If \code{\emph{T}} is \textbf{not} a BitString type, then a \code{init} function
MUST be specified when the \code{Self} type is being constructed. Otherwise, and
if an initialisation function is not specified,
an initialisation function will be generated by the runtime, with
\code{\emph{T}} as a BitString type constructed by the runtime, whose length is
appropriate for the type \code{Self}.

There are two properties that follow from these constraints on \code{init}
functions. Firstly, given that runtime defines only one non-\code{Nothing} type
(\code{DataUnit}), it is always possible to construct a chain from a
\code{DataUnit} (the base type) to any other representable type via a series of
\code{init} functions. Secondly, given that each type has only a single
initialisation function, only one such chain of \code{init} functions exists for
a given representable type. This allows for the unambiguous parsing of incoming
data.

% - - - - - - - - - - - - - - - - - - - - - - - - - - - - - - - - - - - - - - - - - - - - - - - - -
\subsubsection{Primitive types}
\label{sec:representable-primitive}

The runtime defines two primitive representable types:

\begin{itemize}
  \item The \code{Nothing} type is the empty type. It implements the
    \code{Sized} trait and has size 0.

  \item The \code{DataUnit} type is the base type. All data for parsing or
    serialisation is initially represented within the language as this type.
    It is constructed as a \code{BitString} type that has size equal to the
    ``\emph{data\_size}'' \code{Context} field.
\end{itemize}

% - - - - - - - - - - - - - - - - - - - - - - - - - - - - - - - - - - - - - - - - - - - - - - - - -
\subsubsection{Bit Strings}
\label{sec:bit-string}

Representable types of kind \code{BitString} can be constructed to
represent multi-bit values that can be parsed or serialised.

The \code{BitString} type constructor takes as parameters the \emph{name}
of the new bit string type and a \emph{size} in bits. The size may be
set to \code{Nothing}.

Instances of \code{BitString} types implement the \code{Sized} trait. The
\code{size()} method of \code{BitString} types returns \emph{size}, if
\emph{size} is a \code{Number} or an expression that evaluates to a
\code{Number}. If \emph{size} is \code{Nothing}, then the return value of
\code{size()} is inferred by the runtime (see
Section~\ref{sec:runtime-size-inference}).

Instances of \code{BitString} types also implement the \code{Value},
\code{Equality}, and \code{NumberRepresentable} traits.

% - - - - - - - - - - - - - - - - - - - - - - - - - - - - - - - - - - - - - - - - - - - - - - - - -
\subsubsection{Options}
\label{sec:option-type}

Representable types of kind \code{Option} can be constructed to represent
optional types that are either the \code{Nothing} type \emph{or} another
representable type.

The \code{Option} type constructor takes as parameters the \emph{name} of the
new type and another \emph{reference type}. The \code{Option} type is either
\code{Nothing} or the reference type.

Instances of the \code{Option} type implement the \code{TypeEquality} trait.
\code{is\_type(\emph{T})} returns \code{true} if the Option is the type
\code{\emph{T}}, and returns \code{false} otherwise.

Instances of \code{Option} types implement the \code{Sized} trait. \code{size()}
returns an \code{IfElse} expression, where the condition is
\code{This.is\_type(\emph{reference type})}, the \emph{if true} expression is
the size of the reference type, and the \emph{if false} expression is the size
of the \code{Nothing} type.

% - - - - - - - - - - - - - - - - - - - - - - - - - - - - - - - - - - - - - - - - - - - - - - - - -
\subsubsection{Arrays}

Representable types of kind \code{Array} can be constructed to represent
a sequence of elements that can be parsed or serialised.

The \code{Array} type constructor takes as parameters the \emph{name} of
the new array type, the \emph{element type}, the \emph{length} of the
new array, and an \emph{init} function, as described above.
The element type must be a previously defined representable type.
The length is the number of elements contained in the array.
The length may be \code{Nothing}.

Type parameter \code{Self} in the \code{init} function of an \code{Array} refers
to the new type with name \emph{name}.

Instances of the \code{Array} type implement the \code{Sized} trait. The size of
the array is equal to the \emph{length} of the array multiplied by the
\code{size()} of the element type. If \emph{length} is \code{Nothing}, then the
\code{size()} of the array will be inferred by the runtime (see
Section~\ref{sec:runtime-size-inference}).

Instances of \code{Array} types also implement the \code{Equality},
and \code{IndexCollection} traits.
Two arrays are equal if they have the same element type and length, and
their elements are equal.

% - - - - - - - - - - - - - - - - - - - - - - - - - - - - - - - - - - - - - - - - - - - - - - - - -
\subsubsection{Structure Types}

Representable types of kind \code{Struct} can be constructed to represent
a sequence of fields, with possibly different types, that can be parsed
and serialised.

The \code{Struct} type constructor takes as parameters the \emph{name} of
the new structure type, a list of \emph{fields}, a possibly empty list of
\emph{constraints} on those fields, a possibly empty list of
\emph{actions} that are to be carried out once the structure has been
parsed, and an \code{init} function, as described above.

Type parameter \code{Self} in the \code{init} function of a \code{Struct} refers
to the new type with name \emph{name}.

\emph{Fields} are constructed with a \emph{name} for the field, the
\emph{type} of the field, and an expression indicating if the field
\emph{is present} in a particular instantiation of the structure type.
Structure types MUST contain at least one field.

Field names are formed of upper and lower case ASCII letters,
digits, dollar signs, or underscores (\code{A-Za-z0-9\$\_}). The field name
MUST begin with a lower case letter (\code{a-z}).

Field types MUST have been previously defined.
Fields can be of \code{BitString}, array, structure type, or enumerated type,
or a type derived from such a type. The names of each field MUST be unique
within a structure definition, but several fields can have the same type.

Each field has an \emph{is present} expression that indicates
whether that field is present in a particular instantiation of the
structure type. This is used to model data formats that contain optional
fields. Expressions are described in Section \ref{sec:expressions}. The
\emph{is present} expression MUST evaluate to a \code{Boolean} value.

A structure type implements the \code{Sized} trait. The size of a structure is
inferred by the runtime.

A structure type also implements the \code{Equality} trait. Two structures
are equal if they are the same type, and the
corresponding fields are equal. Two fields are equal if they have the same
\emph{name}, the same \emph{type}, and the same \emph{is present} expression
that evaluates to the same value.

A structure type is parameterised by a set of constraints (i.e., boolean
expressions that MUST evaluate to \code{True}) on the
fields. The set of constraints MAY be empty. Expressions are described in
Section \ref{sec:expressions}.

Finally, on the successful parsing of the structure type, the list of \emph{actions} is
evaluated. The list of actions MAY be empty.  The expressions in the list of actions MUST
be tree expressions, and each expression MUST return \code{Nothing}. Actions are typically
expected to update the parsing context (see Section~\ref{sec:context}).

% - - - - - - - - - - - - - - - - - - - - - - - - - - - - - - - - - - - - - - - - - - - - - - - - -
\subsubsection{Enumerated Types}
\label{sec:enum-type}

Representable types of kind \code{Enum} can be constructed to represent
data that can exist as one of several possible variants when parsed or
serialised.

The \code{Enum} type constructor takes as parameters the \emph{name} of the
new \code{Enum} type, an ordered list of the types of the \emph{variants} that
the \code{Enum} might take, and an \emph{init} function, as described above.

Type parameter \code{Self} in the \code{init} function of an \code{Enum} refers
to the new type with name \emph{name}.

Variant types MUST have been
previously defined, and the variants list MUST NOT be empty. Variants of
an enumerated type have their own types, but are otherwise unnamed.
Variants can be of \code{BitString}, array, structure type, or enumerated type, or
a subtype of such a type.

The \code{Enum} will take the type of the first variant type that the data can be
successfully parsed or serialised as. Parsing or serialisation will be attempted for each
variant type in turn until successful, or until the list of variant types has been
exhausted.

An enumerated type implements the \code{TypeEquality} trait.
\code{is\_type(\emph{T})} returns \code{true} if the enumerated type's variant
is of type \code{\emph{T}}, and returns \code{false} otherwise.

An enumerated type implements the \code{Sized} trait.  \code{size()}
returns an \code{IfElse} expression, where the condition is
\code{This.is\_type(\emph{variant})}, the \emph{if true} expression is
the size of the \emph{variant} type, and the \emph{if false} expression is
either a further \code{IfElse} expression conditional upon the next variant in
the list of variants (if there are two or more remaining variants), or the
\code{size()} of the final variant in the list.

% - - - - - - - - - - - - - - - - - - - - - - - - - - - - - - - - - - - - - - - - - - - - - - - - -
\subsubsection{Derived Types}

The type system includes shorthand for the creation of new types from other
types. A derived type has the same representation
and traits as the type it is derived from, but has a new name and is
distinct from the original type. It MAY also implement additional traits.

The type constructor of a derived type is parameterised by the \emph{name} of
the new type, the \emph{type} that it is derived from, a (possibly empty) list
of additional \emph{traits} the new type implements, and an \emph{init}
function, as described above.

Type parameter \code{Self} in the \code{init} function of a derived type refers
to the new type with name \emph{name}.

A new type can be derived from any previously defined representable type. It is
an error to define a new type that has the same name as an existing type. A new
type cannot be derived from itself.

Once constructed, derived types do not have any relationship with the type they
are derived from.

%--------------------------------------------------------------------------------------------------
\subsection{Expressions}
\label{sec:expressions}

Expressions are split into two classes:
tree (\code{MethodInvocation}, \code{FunctionInvocation}, \code{FieldAccess},
      \code{ContextAccess}, and \code{IfElse})
and
leaf (\code{This}, and \code{Constant}).

Expressions MUST evaluate to a result that is of a previously defined type.

% - - - - - - - - - - - - - - - - - - - - - - - - - - - - - - - - - - - - - - - - - - - - - - - - -
\subsubsection{Tree Expressions}

A \code{MethodInvocation} expression is parameterised by a \emph{target} expression, a
\emph{method name}, and list of \emph{arguments} for that method. The expression evaluates
to the result of calling the named method, with \code{self} set to the \emph{target}
expression and the other arguments set as specified. Arguments are specified as expressions.

A \code{FunctionInvocation} expression is parameterised by a \emph{function name} and a list of
\emph{arguments}, specified as expressions. The expression evaluates to the result of
calling the named function with the arguments specified. The \emph{name} MUST refer to
a previously defined \code{Function}. The set
of arguments MUST contain values for all of the parameters specified
in the \code{Function} definition, and the \code{value} expression for each
MUST evaluate to a value matching the type of the parameter as specified in
the \code{Function} definition.

A \code{FieldAccess} expression is parameterised by a \emph{target} expression, and a
\emph{field name}. The \emph{target} expression MUST resolve to a structure type. A
\code{FieldAccess} expression resolves to the type of the \emph{field name} within the
\emph{target} structure.

A \code{ContextAccess} expression is parameterised by a \emph{field name}. The expression
resolves to the type of the \emph{field name} within the protocol's context.
It is an error to evaluate a \code{ContextAccess} expression on a protocol
that protocol does not contain a \code{Context} instance.

A \code{IfElse} expression is parameterised by three expressions: a \emph{condition},
an \emph{if true} expression and an \emph{if false} expression. The expression resolves
to the \code{Boolean} type. It resolves to the true expression if the condition
expression evaluates to True, otherwise, it evaluates to the false expression.

% - - - - - - - - - - - - - - - - - - - - - - - - - - - - - - - - - - - - - - - - - - - - - - - - -
\subsubsection{Leaf Expressions}

Expressions are always evaluated in the context of a containing type.
A \code{This} expression resolves to that containing type.

A \code{Constant} expression is parameterised by a \emph{type name} and a \emph{value}.
The expression evaluates to the specified value, whose type is specified.

%==================================================================================================
\section{Parser Model}
\label{sec:parser-model}

The goal of the intermediate representation described in this document is to
allow for the unambiguous description of the parsing and serialisation of
binary protocols, from which code that supports these processes can be
generated. However, the intermediate representation limits the constraints it
places on the design and architecture of generated code. This allows
code generation backends to produce code that is idiomatic within the target
language or that fulfils other non-functional properties (such as readability or
performance).

\begin{figure}
\begin{lstlisting}[language=Python]
new_type = Nothing()

# try to parse as each listed PDU type in turn
for pdu in protocol.PDUs:
  from_type = type(pdu.init.from)

  # build chain of constructors that begins with DataUnit
  init_chain = []
  while from_type != DataUnit:
    init_chain.append(from_type)
    from_type = type(from_type.init.from)
  input_chain.append(pdu)

  # call constructors in sequence
  new_type = input.copy()
  for init_type in init_chain:
    new_type = init_type.init(new_type)
    if type(new_type) is Nothing:
      continue # try next PDU type if an initialiser returns Nothing
  return new_type
\end{lstlisting}
\caption{Pseudocode for parsing process}
\label{fig:parsing-pseudocode}
\end{figure}

Figure~\ref{fig:parsing-pseudocode} gives pseudocode for the parsing process, as
captured by the intermediary language. Parsing begins with an incoming PDU
(\code{input}). Within the model described by the intermediate representation,
this PDU originates as a \code{DataUnit}. From this \code{DataUnit}, the
parser should then attempt to construct the first PDU type as listed within the
\code{Protocol}. This begins by building the chain of constructors from the
\code{DataUnit} to the PDU type. This chain of constructors is then called in
sequence, until either parsing fails (because a constructor returns
\code{Nothing}) or the PDU type is constructed successfully. If parsing fails,
then the construction of the next PDU type is attempted. This continues until
the list of PDU types is exhausted (in which case the incoming data cannot be
parsed) or a PDU type is successfully constructed.

The intermediary representation does not specify how the generated code should
represent the types described, or their traits and methods. It is only required
that the logical consistency of the intermediary representation's type system
is maintained. For example, the particular representation of our language's
\code{Number} type is not specified; It remains that this type must be
represented in a way that supports the traits and methods that apply to the
\code{Number} type.

%--------------------------------------------------------------------------------------------------
\subsection{Type size inference}
\label{sec:runtime-size-inference}

The constructors for representable types allow for the size of the constructed
types to be unspecified. However, the runtime MUST be able to infer an
expression for the size of all representable types once all types have been
constructed.

The runtime performs a size inference process, starting with each type specified
as a PDU in the \code{Protocol}'s \emph{PDUs} list. Size inference is a
recursive process. We begin by defining the \emph{container size}: that is, the
size of the outer type. This is initially set to \code{data\_size}.

The runtime infers a size for each type as follows:
\begin{description}
  \item[BitString]
    The size of a \code{BitString} type is set to the \emph{container size}.
  \item[Option]
    The size of an \code{Option} type is as defined in
    Section~\ref{sec:option-type}.

    The size inference process continues by inferring the size of the
    \code{Option}'s \emph{reference type}, where the \emph{container size} is
    unchanged.
  \item[Array]
    The size of the array's \emph{element type} is inferred first. Then, the
    array's length can be inferred as \emph{container size} divided by the size
    of each element. Finally, the size of the array is set to
    \emph{container size}.
  \item[Structure]
    A structure can contain at most one field whose type has a size that is
    unspecified, and that must be infered by the runtime. The size of that
    field is infered by the runtime, with \emph{container size} set to its
    existing size less the sum of the sizes of all of the other fields.
  \item[Enumerated]
    The size of an \code{Enum} type is as defined in
    Section~\ref{sec:enum-type}.

    The size inference process continues by inferring the size of each of the
    \code{Enum}'s \emph{variants}, where the \emph{container size} is unchanged.
\end{description}

%==================================================================================================
\section{Outstanding issues}

The language specification does not yet consider:
\begin{itemize}
  \item The role of the \code{init} function for \code{Enum} types.
  \item The ways in which the size inference process can fail.
  \item How the size inference process interacts with constraints.
\end{itemize}

%==================================================================================================
\section{Acknowledgements}

This work was supported by the Engineering and Physical Sciences Research
Council (grant EP/R04144X/1).

%==================================================================================================
\bibliographystyle{abbrv}
\bibliography{ir}
%==================================================================================================
% The following information gets written into the PDF file information:
\ifpdf
  \pdfinfo{
    /Title        (The Glasgow Packet Language: Intermediate Representation and Execution Model)
    /Author       (Stephen McQuistin and Colin Perkins)
    /Subject      (The Glasgow Packet Language)
    /Keywords     (Parsing, Network Protocols, Packet Formats)
    /CreationDate (D:20180727162600Z)
    /ModDate      (D:20180727162600Z)
    /Creator      (LaTeX)
    /Producer     (pdfTeX)
  }
  % Suppress unnecessary metadata, to ensure the PDF generated by pdflatex is
  % identical each time it is built:
  \ifdefined\pdftrailerid
    % The \pdftrailerid and \pdfsuppressptexinfo macros were both introduced
    % in pdfTeX 3.14159265-2.6-1.40.17. If one is present, the other will be.
    \pdftrailerid{}
    \pdfsuppressptexinfo=15
  \fi
\fi
%==================================================================================================
\end{document}
% vim: set ts=2 sw=2 tw=75 et ai:
